\documentclass[11pt]{article}

%%%%%%%%%%%%%%%%%%%%%%%%%%%%%%%%%%%%%%%%%%%%%%%%%%%%%%%%%%%%%%%
% Packages
%%%%%%%%%%%%%%%%%%%%%%%%%%%%%%%%%%%%%%%%%%%%%%%%%%%%%%%%%%%%%%%
\usepackage[margin=1in]{geometry}
\usepackage{graphicx}
\usepackage{pdflscape}
\usepackage{titlepic}
\usepackage{amsthm,amsmath,amssymb}
\usepackage{sansmathfonts}
\usepackage{listings}
\usepackage[shortlabels]{enumitem}
\usepackage{hyperref}
\usepackage{nameref}
\usepackage{wrapfig}
\usepackage{lipsum}
\usepackage{cite}
\usepackage[english]{babel} % English blindtext
\usepackage{blindtext} % macros for producing dummy output
\bibliographystyle{unsrt}
\usepackage{fancyvrb} % nicer verbatim handling
\usepackage{sansmathfonts}
\renewcommand{\familydefault}{\sfdefault}

\usepackage[colorinlistoftodos,textwidth=20mm,shadow]{todonotes}
% \usepackage[disable]{todonotes}
% \usepackage{draftwatermark}
% \SetWatermarkText{\parbox{\textwidth}{\centering DRAFT v0.1}}
% \SetWatermarkScale{1.0}
\usepackage{multicol}

%%%%%%%%%%%%%%%%%%%%%%%%%%%%%%%%%%%%%%%%%%%%%%%%%%%%%%%%%%%%%%%
% Local defines
%%%%%%%%%%%%%%%%%%%%%%%%%%%%%%%%%%%%%%%%%%%%%%%%%%%%%%%%%%%%%%%
\DeclareMathOperator{\sgn}{sgn}
\newcommand{\ith}{\ensuremath{i^\text{th}} }
\newcommand{\jth}{\ensuremath{j^\text{th}} }
\newcommand{\nth}{\ensuremath{n^\text{th}} }
\newcommand{\blditm}[1]{\item{\textbf{#1}}}

% Use \qst command for questions that need answering
\newcommand{\qst}[1]{\todo[color=magenta]{#1}}
% Use \lookup for items that need further research
\newcommand{\lookup}[1]{\todo[color=green]{#1}}
% Use \tsk for tasks that still need to be completed
\newcommand{\tsk}[1]{\todo[inline,color=cyan]{#1}}
% Use \idea for ideas not ready to be in text
\newcommand{\idea}[1]{\todo[inline,color=yellow]{#1}}
% Use \err to highlight were there are serious problems
\newcommand{\err}[1]{\todo[inline,color=red]{#1}}
% use to put a note in the margin
\newcommand{\sidenote}[2]{\todo[color=#1,size=\tiny]{#2}}

\newcommand{\etal}{\textit{et al. }}
\newcommand{\ie}[1]{\textit{i.e.} {#1}}
\newcommand{\eg}[1]{\textit{e.g.} {#1}}
\newcommand{\etc}[1]{\textit{etc.}{#1}}

% Use ``\cite{NEEDED}'' to get Wikipedia-style ``citation needed'' in document
\usepackage{ifthen}
\let\oldcite=\cite
\renewcommand\cite[1]{\ifthenelse{\equal{#1}{NEEDED}}{\textsuperscript{\texttt{\color{red}[citation~needed]}}}{\oldcite{#1}}}

%%%%%%%%%%%%%%%%%%%%%%%%%%%%%%%%%%%%%%%%%%%%%%%%%%%%%%%%%%%%%%%
% Preamble
%%%%%%%%%%%%%%%%%%%%%%%%%%%%%%%%%%%%%%%%%%%%%%%%%%%%%%%%%%%%%%%

\author{Eric Wait}
\title{Future of Image Analysis Software}
\date{\today}

\begin{document}
	\maketitle
	\newpage

	\tableofcontents

	\listoftodos
	\newpage

	% \begin{multicols}{2}
	\section{Main Points}\label{sec:main_points}
		Image analysis has made great strides in terms of capability and rigor.
		However, the current software tools have not kept pace.
		ImageJ has historically been the best go-to software tool for microscopist.
		All microscope data should go through five stages (Figure\ref{fig:overview}): initial data collection and handling, visualization, preprocessing, processing, summarization, and finally sharing.
		Inefficiencies at any of these stages can render data unusable.
		Insufficient data handling can be the main bottle neck of any image processing pipeline.
		At worst, data handling strategies can lead to data loss.
		Without appropriate visualization tools, data cannot be triage during collection or shortly thereafter.
		This leads to over/under collection of useable data.
		Processing can be split into two separate sections, preprocessing \ie{``cleaning'' the data} and phenomena detection.
		Some imaging techniques require some sort of reconstruction to produce useable output.
		And no collection system is without artifacts.
		This means that all data can benefit from preprocessing.
		With data that is ``cleaned,'' other techniques are needed to identify the phenomena being studied.
		One example of 
		The following sections will detail each of these stages in detail. 
		
		\begin{figure}
			\centering
				\includegraphics[width=\textwidth]{figures/overview.pdf}
			\label{fig:overview}
		\end{figure}
	% \end{Main Points}

	\section{Data Storage}\label{sec:data_storage}
		Datasets are getting large and unwieldy.
		Modern pipelines are are complex and can be hindered by I/O.
		Data storage should emphasize the following:
		\begin{itemize}
			\item Octree
			\item Block based losseless compression \cite{Amat2015}
			\item GPU acceleration of Octree creation
			\item Stand alone conversion
			\item Operating System Integration
			\item Looks like a single file to the user (utilites to pull out subset...)
			\item Distributed / Parallel Access \cite{Amat2015}
			\item Non-contigious disc space (\emph{this will take the most work})
			\item Optimal network transfer
			\item Optimized for Instant Visualization \cite{Wait2014}
			\item Fast Preprocessing \cite{Wait2019}
			\item Fast AI/ML/Network processing
		\end{itemize}

		\subsection{History / Current Formats}
			\begin{itemize}
				\item tiff files (\href{https://www.loc.gov/preservation/digital/formats/fdd/fdd000022.shtml}{Tiff specs})
				\item DICOM files (\href{https://www.dicomstandard.org/current/}{DICOM homepage})
				\item OME Bioformats (\href{https://docs.openmicroscopy.org/bio-formats/6.3.1/about/index.html}{Mission})
				\item OME Pryamid format (\href{https://docs.openmicroscopy.org/ome-model/6.0.1/omero-pyramid/index.html}{link})
				\item 
			\end{itemize}
	% \end{Data Storage}
	
	\section{Visualization}\label{sec:visualization}
		Visualization is often done on a case by case manner or for a specific need \cite{NEEDED}.
		The method of visualization can hinder insight.
		There are many visualization techniques written for individual niches \cite{NEEDED}.
		Modern visualization needs to be:
		\begin{itemize}
			\item Simple interface (Intuitive)
			\item Discoverable operations
			\item Fast
			\item Informative
			\item Customizable
			\item High Quality
			\item Physically Based
			\item Easy to make figures and movies
			\item Reproducible
			\item Reporting
			\item[\vspace{\fill}]
		\end{itemize}
	% \end{Visualization}

	\section{Preprocessing}\label{sec:preprocessing}
		Most data needs some preprocessing to remove background or clean up unwanted signal \cite{Aaron2019}.
		There are a variety of new methods that are yet to be implemented in general software \cite{NEEDED}.
		Preprocessing can be time consuming on large datasets.
		\tsk{Hydra graph here?}
		Many filters have not been efficiently implemented for 3-D or distributed architecture (GPU, local cluster, cloud computing, \etc).
		Edge artifacts are an increasing issue with anisotropic data (\eg{confocal}).
	% \end{Preprocessing}\label{sec:preprocessing}

	\section{Segmentation, Tracking, Connected Component Analysis}\label{sec:seg_track_cc_analysis}
	% \end{Segmentation, Tracking, Connected Component Analysis}\label{sec:seg_track_cc_analysis}
	
	\section{Summarization / High-level Analysis}\label{sec:sum_highlevel_analysis}
	% \end{Summarization / High-level Analysis}\label{sec:sum_highlevel_analysis}
	
	\section{Validation / Correction}\label{sec:validation_correction}
	% \end{Validation / Correction}\label{sec:validation_correction}
	
	\section{Bulk Analysis}\label{sec:bulk_analysis}
	% \end{Bulk Analysis}\label{sec:bulk_analysis}
	
	\section{Reporting}\label{sec:reporting}
		Filters should be well documented to include the references that first introduced the technique.
		Processing pipeline should produce a report that includes references, order, and parameters.

	% \end{Reporting}\label{sec:reporting}

	% \end{multicols}

	\newpage
	\bibliography{ref}
\end{document}
